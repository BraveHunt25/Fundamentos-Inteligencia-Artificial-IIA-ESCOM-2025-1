% !TeX root = main.tex
\section{Un subsistema de navegacion de un robot contiene un conjunto de 9 sensores infrarojos (cada uno regresa un 1 si detecta la presencia de algo justo enfrente) colocados a modo de matriz de 3 x 3. se requiere que dicho subsistema regrese la deteccion de obstaculos en alguna direccion de acuerdo a lo siguiente:}
\subsection{Arriba obstaculo total: cuando los 3 sensores de la parte superior detectan algo.}
Definimos la matriz y las celdas como se presenta a continuación:
\begin{table}[!ht]
    \centering
    \begin{tabular}{|c|c|c|}
        \hline
        a & b & c \\
        \hline
        d & e & f \\
        \hline
        g & h & i \\
        \hline
    \end{tabular}
    \caption{Matriz de sensores}\label{table:matriz-sensores}
\end{table}
Con esto, para registrar que hay un obstáculo total en la parte superior, se considera que las entradas de interés son la $a$, $b$, y $c$. El resto de entradas no influyen en la detección de obstáculos total en la parte de arriba. La tabla de verdad es:

\begin{table}[!ht]
    \centering
    \begin{tabular}{|c|c|c|c|}
        \hline
        a & b & c & $\text{Arriba}_{\text{TOTAL}}$\\
        \hline
        0 & 0 & 0 & 0 \\
        \hline
        0 & 0 & 1 & 0 \\
        \hline
        0 & 1 & 0 & 0 \\
        \hline
        0 & 1 & 1 & 0 \\
        \hline
        1 & 0 & 0 & 0 \\
        \hline
        1 & 0 & 1 & 0 \\
        \hline
        1 & 1 & 0 & 0 \\
        \hline
        1 & 1 & 1 & 1 \\
        \hline
    \end{tabular}
    \caption{Tabla de verdad para los obstáculos totales en la parte de arriba}\label{table:arriba-total}
\end{table}

La expresión de salida es:
\begin{equation*}
    a \wedge b \wedge c
\end{equation*}
\subsection{Arriba obstaculo parcial: cuando alguno de los 3 sensores de la parte superior detecta algo}
Por otro lado, para los parciales, solo es necesario que alguno de los 3 se active, por lo que su tabla de verdad es:
\begin{table}[!ht]
    \centering
    \begin{tabular}{|c|c|c|c|}
        \hline
        a & b & c & $\text{Arriba}_{\text{parcial}}$\\
        \hline
        0 & 0 & 0 & 0 \\
        \hline
        0 & 0 & 1 & 1 \\
        \hline
        0 & 1 & 0 & 1 \\
        \hline
        0 & 1 & 1 & 1 \\
        \hline
        1 & 0 & 0 & 1 \\
        \hline
        1 & 0 & 1 & 1 \\
        \hline
        1 & 1 & 0 & 1 \\
        \hline
        1 & 1 & 1 & 1 \\
        \hline
    \end{tabular}
    \caption{Tabla de verdad para los obstáculos parciales en la parte de arriba}\label{table:arriba-parcial}
\end{table}

\vspace{1cm}
La expresión de salida es:
\begin{equation*}
    a \vee b \vee c
\end{equation*}

\vspace{2cm}
\subsection{De forma similar con los sensores correspondientes para las direcciones de abajo, izquierda y derecha.}
Dado que se tratan de forma similar, a continuación se presentarán las expresiones lógicas con las entradas correspondientes para cada señal de obstáculos:
\begin{table}[!ht]
    \centering
    \begin{tabular}{|c|c|c|c|c|}
        \hline
        $\text{Entrada}_1$ & $\text{Entrada}_2$ & $\text{Entrada}_3$ & Expresión & Obstáculo \\
        \hline
        a & b & c & $a \wedge b \wedge c$ & $\text{Arriba}_{\text{TOTAL}}$ \\
        \hline
        a & b & c & $a \vee b \vee c$ & $\text{Arriba}_{\text{parcial}}$ \\
        \hline
        g & h & i & $g \wedge h \wedge i$ & $\text{Abajo}_{\text{TOTAL}}$ \\
        \hline
        g & h & i & $g \vee h \vee i$ & $\text{Abajo}_{\text{parcial}}$ \\
        \hline
        a & d & g & $a \wedge d \wedge g$ & $\text{Izquierda}_{\text{TOTAL}}$ \\
        \hline
        a & d & g & $a \vee d \vee g$ & $\text{Izquierda}_{\text{parcial}}$ \\
        \hline
        c & f & i & $c \wedge f \wedge i$ & $\text{Derecha}_{\text{TOTAL}}$ \\
        \hline
        c & f & i & $c \vee f \vee i$ & $\text{Derecha}_{\text{parcial}}$ \\
        \hline
    \end{tabular}
\end{table}

\textbf{\Large Un caso particular es determinar la direccion frontal, esto se determina de la siguiente forma:}
\subsection{Frontal obstáculo parcial: cuando existe algun obstáculo parcial en alguna direccion y ademas el sensor de enmedio detecta algo}

Para esto, traducimos la sentencia a lógica proposicional:

\begin{equation*}
    \text{Frontal}_{\text{parcial}} = e \wedge (\text{Arriba}_{\text{parcial}} \vee \text{Abajo}_{\text{parcial}} \vee \text{Izquierda}_{\text{parcial}} \vee \text{Derecha}_{\text{parcial}})
\end{equation*}

Si desarrollamos y simplificamos las variables que se repiten, es equivalente a:
\begin{equation*}
    e \wedge (a \vee b \vee c \vee d \vee f \vee g \vee h \vee i)
\end{equation*}

\subsection{Frontal obstaculo total: cuando existe un obstaculo total en todas las direcciones y ademas el sensor de enmedio detecta algo}
Para esto, traducimos la sentencia a lógica proposicional:
\begin{equation*}
    \text{Frontal}_{\text{TOTAL}} = e \wedge (\text{Arriba}_{\text{TOTAL}} \wedge \text{Abajo}_{\text{TOTAL}} \wedge \text{Izquierda}_{\text{TOTAL}} \wedge \text{Derecha}_{\text{TOTAL}})
\end{equation*}

Si desarrollamos y simplificamos las variables que se repiten, es equivalente a:
\begin{gather*}
    e \wedge (a \wedge b \wedge c \wedge d \wedge f \wedge g \wedge h \wedge i) \\
    e \wedge a \wedge b \wedge c \wedge d \wedge f \wedge g \wedge h \wedge i
\end{gather*}