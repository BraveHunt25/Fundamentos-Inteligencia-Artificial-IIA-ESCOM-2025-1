% !TeX root = main.tex
\section{Una máquina de producción tiene sensores que indican la existencia de:}
\begin{itemize}
    \item \large Tornillos
    \item \large Tuercas
    \item \large Rondana
    \item \large Remaches
\end{itemize}
\subsection{Modo A:operación normal. Cuando existen tornillos, tuercas y rondanas, sin importar si hay remaches.}
\begin{table}[!ht]
    \centering
    \begin{tabular}{|c|c|c|c|c|}
        \hline
        a & b & c & d & Modo A \\
        \hline
        0 & 0 & 0 & 0 & 0 \\
        \hline
        0 & 0 & 0 & 1 & 0 \\
        \hline
        0 & 0 & 1 & 0 & 0 \\
        \hline
        0 & 0 & 1 & 1 & 0 \\
        \hline
        0 & 1 & 0 & 0 & 0 \\
        \hline
        0 & 1 & 0 & 1 & 0 \\
        \hline
        0 & 1 & 1 & 0 & 0 \\
        \hline
        0 & 1 & 1 & 1 & 0 \\
        \hline
        1 & 0 & 0 & 0 & 0 \\
        \hline
        1 & 0 & 0 & 1 & 0 \\
        \hline
        1 & 0 & 1 & 0 & 0 \\
        \hline
        1 & 0 & 1 & 1 & 0 \\
        \hline
        1 & 1 & 0 & 0 & 0 \\
        \hline
        1 & 1 & 0 & 1 & 0 \\
        \hline
        1 & 1 & 1 & 0 & 1 \\
        \hline
        1 & 1 & 1 & 1 & 1 \\
        \hline        
    \end{tabular}
    \caption{Tabla de verdad para modo A}\label{table:modo-A}
\end{table}
\newpage
\subsection{Modo B. Operación con remaches. Cuando faltan ya sea tornillos, tuercas o rondanas y hay remaches}
\begin{table}[!ht]
    \centering
    \begin{tabular}{|c|c|c|c|c|}
        \hline
        a & b & c & d & Modo B \\
        \hline
        0 & 0 & 0 & 0 & 0 \\
        \hline
        0 & 0 & 0 & 1 & 1 \\
        \hline
        0 & 0 & 1 & 0 & 0 \\
        \hline
        0 & 0 & 1 & 1 & 1 \\
        \hline
        0 & 1 & 0 & 0 & 0 \\
        \hline
        0 & 1 & 0 & 1 & 1 \\
        \hline
        0 & 1 & 1 & 0 & 0 \\
        \hline
        0 & 1 & 1 & 1 & 1 \\
        \hline
        1 & 0 & 0 & 0 & 0 \\
        \hline
        1 & 0 & 0 & 1 & 1 \\
        \hline
        1 & 0 & 1 & 0 & 0 \\
        \hline
        1 & 0 & 1 & 1 & 1 \\
        \hline
        1 & 1 & 0 & 0 & 0 \\
        \hline
        1 & 1 & 0 & 1 & 1 \\
        \hline
        1 & 1 & 1 & 0 & 0 \\
        \hline
        1 & 1 & 1 & 1 & 0 \\
        \hline        
    \end{tabular}
    \caption{Tabla de verdad para modo B}\label{table:modo-B}
\end{table}
\newpage
\subsection{Modo C:alto total, cuando no existe ninguno de los 4}
\begin{table}[!ht]
    \centering
    \begin{tabular}{|c|c|c|c|c|}
        \hline
        a & b & c & d & Modo C \\
        \hline
        0 & 0 & 0 & 0 & 1 \\
        \hline
        0 & 0 & 0 & 1 & 0 \\
        \hline
        0 & 0 & 1 & 0 & 0 \\
        \hline
        0 & 0 & 1 & 1 & 0 \\
        \hline
        0 & 1 & 0 & 0 & 0 \\
        \hline
        0 & 1 & 0 & 1 & 0 \\
        \hline
        0 & 1 & 1 & 0 & 0 \\
        \hline
        0 & 1 & 1 & 1 & 0 \\
        \hline
        1 & 0 & 0 & 0 & 0 \\
        \hline
        1 & 0 & 0 & 1 & 0 \\
        \hline
        1 & 0 & 1 & 0 & 0 \\
        \hline
        1 & 0 & 1 & 1 & 0 \\
        \hline
        1 & 1 & 0 & 0 & 0 \\
        \hline
        1 & 1 & 0 & 1 & 0 \\
        \hline
        1 & 1 & 1 & 0 & 0 \\
        \hline
        1 & 1 & 1 & 1 & 0 \\
        \hline        
    \end{tabular}
    \caption{Tabla de verdad para modo C}\label{table:modo-C}
\end{table}